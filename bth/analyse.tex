\chapter{Analyse}
\label{cha:analyse}
In diesem Kapitel werden die in der Einleitung genannten Schwerpunkte analysiert.
Dazu werden verschiedene \glspl{pp} betrachtet. Auf Basis der Analyse dieser \glspl{pp} werden dann im nächsten Schritt Konzepte entwickelt.

Zusätzlich zu diesen \glspl{pp} wird analysiert, wie hoch der Aufwand für ein Update auf höhere Versionen des Tools ist. 

Am Ende des Abschnitts der Aufwandseinschätzung, werden die einzelnen Versionen unter folgenden Punkten miteinander verglichen.

\subsubsection{Integration des \ac{VCS}}
Abseits der betrachteten Punkte wird die Integration des Tools in einem Version-Control-System (\ac{VCS}) vorgestellt. Die Vorteile der Nutzung sollen dadurch gezeigt werden.\\
Ein Vorteil ist, dass Änderungen von mehreren Mitarbeitenden durchgeführt und nachvollzogen werden können. Zusätzlich entsteht so ein Überblick über die bereits durchgeführten Änderungen und wer diese durchgeführt hat. Dies senkt den Zeitaufwand für zukünftige Analysen der Implementierung des Tools.

\newpage