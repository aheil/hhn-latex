%Dokumentklasse
\documentclass[a4paper,14pt]{scrreprt}
\usepackage[left= 2.5cm,right = 2cm, bottom = 4 cm]{geometry}
%\usepackage[onehalfspacing]{setspace}
% ============= Packages =============





% Dokumentinformationen
\usepackage[
	pdftitle={Titel der Abschlussarbeit},
	pdfsubject={},
	pdfauthor={Euer Name},
	pdfkeywords={},	
	%Links nicht einrahmen
	hidelinks
]{hyperref}


% Standard Packages
\usepackage[utf8]{inputenc}

\usepackage{nameref}
\usepackage[ngerman]{babel}
\usepackage[T1]{fontenc}
\usepackage{graphicx, subfig}
\usepackage[export]{adjustbox}
%Use svgs- couldnt find out how it works
\usepackage{svg}
\usepackage{float}
%Glossary
\usepackage{booktabs}
\graphicspath{{img/}}
\usepackage{fancyhdr}
\usepackage{lmodern}
\usepackage{color}
\usepackage{acronym}
\usepackage{listings}
\lstset{
basicstyle=\footnotesize\ttfamily,
% specifies that lines shouldn't overrun the margin
breaklines=true
% specified that, if possible, lines should be broken at whitespace
breakatwhitespace=true 
}

%Use todo notes in text
\usepackage{todonotes}


% ============= BibLatex =============
\usepackage[backend=bibtex,style=ieee]{biblatex}
%Literaturverzeichnis
\addbibresource{Literatur.bib}


% zusätzliche Schriftzeichen der American Mathematical Society
\usepackage{amsfonts}
\usepackage{amsmath}

%nicht einrücken nach Absatz
%\setlength{\parindent}{0pt}



%for glossar usage
\usepackage[toc]{glossaries}
\loadglsentries{glossaries.tex}

% ============= Kopf- und Fußzeile =============
\pagestyle{fancy}
%
\lhead{}
\chead{}
\rhead{\slshape \leftmark}
%%
\lfoot{}
\cfoot{\thepage}
\rfoot{}
%%
\renewcommand{\headrulewidth}{0.4pt}
\renewcommand{\footrulewidth}{0pt}

% ============= Package Einstellungen & Sonstiges ============= 
%Besondere Trennungen
\hyphenation{De-zi-mal-tren-nung I-ner-tial-sen-so-rik Skript-ab-lauf}
%Damit das Abkürzungsverzeichnis usw. automatisch hinzugefügt wird zum ToC
\usepackage[nottoc]{tocbibind}
\setcounter{secnumdepth}{3}
\setcounter{tocdepth}{3}

         

% ============= Dokumentbeginn =============

\begin{document}
	\emergencystretch 3em 
%Seiten ohne Kopf- und Fußzeile sowie Seitenzahl
\pagestyle{empty}

\begin{center}
\begin{tabular}{p{\textwidth}}



\includegraphics[scale=0.1,right]{img/HHN_Logo.jpg}


\\

\begin{center}
\LARGE{\textsc{
Weiterentwicklung einer \\
lokalen \& globalen \\
Anwendung\\
}}
\end{center}

\\


\begin{center}
\large{\textbf{Software-Engineering}\\}
Fakultät für Informatik \\
Hochschule Heilbronn Campus Sontheim \\
\end{center}
\\

\begin{center}
\textbf{\Large{Bachelor-Thesis}}
\end{center}


\begin{center}
zur Erlangung des akademischen Grades\\
Bachelor of Science
\end{center}


\begin{center}
vorgelegt von
\end{center}

\begin{center}
\large{\textbf{Dein Name}} \\
\small{Matrikelnr.: 123456} \\
\end{center}

\begin{center}
\large{Datum der Abgabe}
\end{center}

\\

\begin{center}
\begin{tabular}{lll}
\textbf{Erstprüfer:} & & Prof an der HS\\
\textbf{Zweitprüfer:} & & Betreuung im Unternehmen\\
\end{tabular}
\end{center}
\end{tabular}
\end{center}

% Beendet eine Seite und erzwingt auf den nachfolgenden Seiten die Ausgabe aller Gleitobjekte (z.B. Abbildungen), die bislang definiert, aber noch nicht ausgegeben wurden. Dieser Befehl fügt, falls nötig, eine leere Seite ein, sodaß die nächste Seite nach den Gleitobjekten eine ungerade Seitennummer hat. 
\cleardoubleoddpage

% pagestyle für gesamtes Dokument aktivieren
\pagestyle{fancy}


% Danksagung
\include{danksagung}

%Inhaltsverzeichnis
\tableofcontents

%Verzeichnis aller Bilder
\listoffigures

%Verzeichnis aller Tabellen
\listoftables

%Abkürzungsverzeichnis
\addchap{Abkürzungsverzeichnis}
\begin{acronym}
	\acro{VCS}{Version Control System}
	\acro{VMPS}{Vehicle Motion Position Sensor}
\end{acronym}
\newpage

\include{einleitung}
\include{grundlagen}

\chapter{Analyse}
\label{cha:analyse}
In diesem Kapitel werden die in der Einleitung genannten Schwerpunkte analysiert.
Dazu werden verschiedene \glspl{pp} betrachtet. Auf Basis der Analyse dieser \glspl{pp} werden dann im nächsten Schritt Konzepte entwickelt.

Zusätzlich zu diesen \glspl{pp} wird analysiert, wie hoch der Aufwand für ein Update auf höhere Versionen des Tools ist. 

Am Ende des Abschnitts der Aufwandseinschätzung, werden die einzelnen Versionen unter folgenden Punkten miteinander verglichen.

\subsubsection{Integration des \ac{VCS}}
Abseits der betrachteten Punkte wird die Integration des Tools in einem Version-Control-System (\ac{VCS}) vorgestellt. Die Vorteile der Nutzung sollen dadurch gezeigt werden.\\
Ein Vorteil ist, dass Änderungen von mehreren Mitarbeitenden durchgeführt und nachvollzogen werden können. Zusätzlich entsteht so ein Überblick über die bereits durchgeführten Änderungen und wer diese durchgeführt hat. Dies senkt den Zeitaufwand für zukünftige Analysen der Implementierung des Tools.

\newpage

\chapter{Konzeptionierung}
\label{cha:concept}


\chapter{Diskussion und Ausblick}
\label{cha: diskussion}

\newpage

%\printbibliography[title=Literaturverzeichnis]
%\printglossary[title = Glossar]
\chapter*{Anhang: Code für die Diagramme}





\chapter*{Eidesstattliche Erklärung}
Hiermit erkläre ich eidesstattlich, dass die vorliegende Arbeit von mir selbstständig und ohne unerlaubte Hilfe angefertigt wurde, insbesondere, dass ich alle Stellen, die wörtlich oder annähernd wörtlich oder dem Gedanken nach aus Veröffentlichungen und unveröffentlichten Unterlagen und Gesprächen entnommen worden sind, als solche an den entsprechenden Stellen innerhalb der Arbeit durch Zitate kenntlich gemacht habe, wobei in den Zitaten jeweils der Umfang der entnommenen Originalzitate kenntlich gemacht wurde. Ich bin mir bewusst, dass eine falsche Versicherung rechtliche Folgen haben wird.
\vspace{3cm}

\begin{tabular}{p{0.45\textwidth}cp{0.45\textwidth}}
	\cline{1-1} \cline{3-3} \\
	\raggedright Ort, Datum & & \raggedright Unterschrift 
\end{tabular}















\end{document}
